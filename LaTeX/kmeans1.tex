\documentclass{article}
\usepackage[utf8]{inputenc}
\usepackage[T2A]{fontenc}
\usepackage[english, russian]{babel}

\begin{document}

Алгоритм K-Means - алгоритм машинного обучения, который используется для группировки объектов на основе их признаков и является одним из наиболее популярных методов кластеризации.

Основная идея метода заключается в следующем:
\begin{enumerate}
\item Задается количество кластеров $K$.
\item Инициализируются $K$ случайных центров каждого кластера.
\item Пока не будет выполнено условие остановки:
\begin{enumerate}
\item Для каждого объекта данных $x$:
\begin{enumerate}
\item Находится ближайший центр кластера с помощью заданной функции расстояния.
\item Объект $x$ присваивается к ближайшему кластеру.
\end{enumerate}
\item Для каждого кластера $k$:
\begin{enumerate}
\item Вычисляется новый центр кластера, как среднее значение всех объектов, отнесенных к данному кластеру.
\end{enumerate}
\end{enumerate}
\item Возвращаются кластеры.
\end{enumerate}

Полученные кластеры описываются векторами средних значений - центроидами. В процессе разбиения выполняется итеративная минимизация внутриклассовых расстояний $J$. Соответствующая целевая функция выглядит следующим образом:

\begin{equation}
J = \sum_{j=1}^{K}  \sum_{x^{(i)} \in C^{(j)}} ||x^{(i)}-c^{(j)}||^2 \to min,
\end{equation}

где $x^{(i)}$ – вектор характеристик объекта, $K$ - количество кластеров, $c^{(j)}$ – центроид кластера $C^{(j)}$.  
Функция расстояния обычно выбирается в зависимости от пространства, в котором расположены объекты. В качетсве используемой метрики рассмотрим Евклидово расстояние:
\begin{equation}
\rho(x_{i};y_{i}) = \sqrt{\sum_{i=1}^{n} (x_{i}-y_{i})^2}.
\end{equation}
Оно представляет собой расстояниме между точками в $n$-мерном пространстве. 


Данный алгоритм может быть применен для сегментации изображений - разбиения исходного изображения на группы пикселей, сегменты, каждый из которых содержит объекты одного типа или имеет одинаковые характеристики.
Рассмотрим работу представленного метода на примере растового изображения. В качетсве объектов $x$ будут выступать пиксели изображения, а в качестве характеристик - их цвета в трехмерном пространсве RGB. 

\begin{enumerate}
\item Мы преобразуем каждое изображение в матрицу признаков. В нашем случае для каждого изображения получаем матрицу размерности $(N \times M \times 3)$, где $N\times M$ - размер изображения, а каждый элемент матрицы содержит тройку значений $R$, $G$ и $B$ для соответствующего пикселя.
\item Затем мы применяем алгоритм K-Means к полученным данным, указав $K$ кластеров и выбрав их центры случайным образом.
\item После достижения сходимости мы можем использовать полученные центры кластеров для создания сегментированного изображения, где каждый пиксель заменяется на центр соответствующего кластера, что и дает нам $K$ сегментов изображения.
\end{enumerate}

Преимуществом использования алгоритма K-Means для сегментации изображений является его простота и эффективность в работе с большими объемами данных. Однако, как и любой алгоритм кластеризации, K-Means не всегда может давать оптимальный результат, особенно если изображение имеет сложную структуру или содержит неоднородные объекты. На практике может быть использовано несколько модификаций и улучшений, таких как использование случайной инициализации центров кластеров несколько раз для более надежных результатов или применение критериев остановки на основе изменения инерции кластеров.

\end{document}